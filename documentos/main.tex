\documentclass[a4paper,12pt]{article}
\usepackage[utf8]{inputenc}
\usepackage[brazil]{babel}
\usepackage{hyperref}

\title{Documentação do projeto JSBridge}
\author{Fabio V. \and Nome Autor2 \and Nome Autor3}
\date{\today}

\begin{document}

\maketitle

\begin{abstract}
Este documento descreve o projeto final de Teoria da Computação e Compiladores, incluindo detalhes sobre suas funcionalidades, uso e arquitetura do projeto.
\end{abstract}

\tableofcontents
\newpage

\section{Introdução}

JsBridge implementa um compilador que transforma código escrito em uma linguagem de entrada específica em linguagem de saída, apresentando diferentes níveis de análise. O compilador inclui funcionalidades como análise sintática, geração de linguagem intermediária e conversão para a linguagem final. A interface web permite visualizar e copiar o código em diferentes estágios do processo de compilação.

\section{Objetivos}

Desenvolver um compilador em ambiente web, que permite transformar código escrito em uma linguagem de entrada específica para uma linguagem de saída final. O projeto visa aplicar conceitos de Teoria da Computação e Compiladores, proporcionando uma ferramenta educacional que facilita a compreensão do processo de compilação. A interface web interativa exibe o código em diferentes estágios, desde a análise sintática até a geração da linguagem final, aprimorando a experiência de aprendizado dos usuários.

\section{Funcionalidades Principais}

JSBridge possui várias funcionalidades chave que facilitam o aprendizado e a experimentação:

\begin{itemize}
\item \textbf{Análise Léxica}: Esta etapa identifica e categoriza os tokens presentes no código fonte, transformando a sequência de caracteres em unidades léxicas significativas.
\item \textbf{Análise Sintática}: Utilizando gramáticas formais, esta fase verifica a estrutura do código fonte para garantir que ele esteja sintaticamente correto.
\item \textbf{Análise Semântica}: Verifica a correção semântica do código fonte, assegurando que as instruções façam sentido lógico dentro do contexto do programa.
\item \textbf{Geração de Código}: Traduz os tokens para linguagem intermediária e então para linguagem final (no caso, javaScript).
\item \textbf{Ambiente de Teste}: Inclui ferramentas e exemplos que permitem aos usuários testar e validar suas implementações, promovendo um entendimento prático dos conceitos teóricos.
\end{itemize}

\section{Usabilidade}

JSBridge foi projetado com foco na usabilidade, proporcionando uma interface web intuitiva para facilitar o processo de compilação. Os usuários podem inserir o código fonte e visualizar as etapas de análise léxica, sintática e semântica, bem como a geração de código intermediário e final. A interface interativa permite a alternância entre essas fases, oferecendo uma visualização clara e organizada do fluxo de compilação, o que torna o aprendizado e a utilização mais acessíveis e eficazes para estudantes e desenvolvedores.

\section{Arquitetura do Projeto}

A arquitetura do projeto \textit{JSBridge} é modular e bem organizada, facilitando a manutenção e a extensão do código. A seguir, uma visão geral dos principais componentes:

\begin{itemize}
\item \textbf{Interface}: O arquivo \texttt{editor.js} É responsável por atribuir funções aos elementos visuais do sistema, em adição, a api \texttt{taboverride} foi utilizada para implementar um ambiente de texto parecido com uma IDE.
\item \textbf{Inicialização e Gestão}: O arquivo \texttt{main.js} Inicializa o sistema, responsável por obter o input do usuário e chamar os métodos das outras classes.
\item \textbf{Linguagem}: O arquivo \texttt{linguagem.js} armazena informações e expressões aceitas pela linguagem.
\item \textbf{Análise léxica}: O arquivo \texttt{lexer.js} executa a análise léxica do código e retorna, se válido, uma árvore com os tokens. 
\item \textbf{Análise Semântica}: O arquivo \texttt{semantic.js} realiza a análise semântica, verificando a coerência lógica do código.
\item \textbf{Centralização das Análises}: O arquivo \texttt{parser.js} centraliza a lógica de análise léxica, sintática e semântica, verificando tokens e estruturas do código fonte.
\item \textbf{Geração de Código intermediário}: O arquivo \texttt{intermediate.js} gera o código intermediário utilizando algoritmos de substituição de termos da cadeia de tokens gerada pelo código do usuário.
\item \textbf{Geração de Código final}: O arquivo \texttt{codigoFinal.js} gera o código final utilizando algoritmos de substituição de termos do código intermediário.
\end{itemize}

\section{Conclusão}

\textit{JSBridge} é uma excelente ferramenta educacional que oferece uma visão aprofundada do processo de compilação. Com suas funcionalidades abrangentes, foco no aprendizado prático e uma arquitetura bem organizada.

\end{document}
